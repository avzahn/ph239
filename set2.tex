\documentclass[12pt]{article}
\usepackage[utf8]{inputenc}
\usepackage[margin=0.6in]{geometry}
\title{}
\author{}
\date{}

\begin{document}

\maketitle

\section{Electric car solar panels}

A typical commute might be ten miles or fifteen kilometers. Car batteries seem to sell for about ten cents per amp hour.


\section{Personal Heating}

Throughout, let's approximate the human foot as equivalent to a liter of water that is ten kelvin below body temperature. The minimum energy we need to spend is 40 kJ.

\subsection{Hot Water}

The sink's volume is probably five liters, and is initially near room temperature at 290 K. Heating it to a body temperature of 310 K requires at minimum 400 kJ. We might consider adding 40 kJ to the minimum value for heat lost to the foot, which might need to be replaced if we want to keep the water at an even temperature.

Supposing the home's water heater is only only 50\% efficient, the cost rises to 800 kJ. Convective and evaporative heat loss to the environment might contribute something too over the several minutes we might expect this process to take. 

So we can say this process takes beteen .5 and 1 MJ and is 5-10\% efficient.

\subsection{Space heater}

Let's say this process takes three minutes. Our 1.5 kW space heater consumes 300 kJ in this time, so the space heater is maybe 13\% efficient.

\subsection{Heating pad}

This might take half an hour, consuming 90 kJ, making it nearly 50\% efficient.

\subsection{Blankets}

Suppose this takes one hour. A hundred watt metabolism will consume 360 kJ, and will be a little better than 10\% efficient.

\end{document}