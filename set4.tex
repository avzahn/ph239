\documentclass[12pt]{article}
\usepackage[utf8]{inputenc}
\usepackage[margin=0.6in]{geometry}
\title{Ph 239 Problem Set 4}
\author{Alex Zahn}
\date{4/29/2016}




\begin{document}

\maketitle

\newcommand{\wmsq}{W/\(\mathrm{m}^2\,\)}
\newcommand{\msq}{\(\mathrm{m}^2\,\)}
\newcommand{\micron}{\(\mu\mathrm{m}\)\,}
\newcommand{\mcb}{\(\mathrm{m}^3\,\)}

\section{Bird Migration}

\subsection{Worst case energy requirements}

It's probably safe to put our theoretical bird in the lift force equals drag force regime. 


\section{Raindrop Size}

Lacking much intuition for the mechanics of raindrop collapse, let's try dimensional analysis on the surface tension and the intertial drag at terminal velocity. The simplest collapse condition we can propose is 

\[ \gamma r = F_{d,term}
\]

where \(r\) is some characteristic length scale describing the largest raindrop. Taking the raindrop to be spherical, \(F_{d,term} = mg = \frac{4}{3}\pi r^3 \rho g\), and

\[ r = \sqrt{\frac{3\gamma}{4\pi\rho g}} \approx 1.3 \, \mathrm{mm}
\]

This seems reasonably in line with personal experience, although the internet claims giant 4-5 mm raindrops have been observed during tropospheric research flights. A little more digging reveals that raindrop theorists (apparently a real thing!) believe typical raindrop size to be formation statistics limited, rather than ram pressure limited.

So this estimate is not especially convincing. 



\end{document}