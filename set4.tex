\documentclass[12pt]{article}
\usepackage[utf8]{inputenc}
\usepackage[margin=0.6in]{geometry}
\title{Ph 239 Problem Set 4}
\author{Alex Zahn}
\date{4/29/2016}




\begin{document}

\maketitle

\newcommand{\wmsq}{W/\(\mathrm{m}^2\,\)}
\newcommand{\msq}{\(\mathrm{m}^2\,\)}
\newcommand{\micron}{\(\mu\mathrm{m}\)\,}
\newcommand{\mcb}{\(\mathrm{m}^3\,\)}

\section{Raindrop Size}

Lacking much intuition for the mechanics of raindrop collapse, let's try dimensional analysis on the surface tension and the intertial drag at terminal velocity. The simplest collapse condition we can propose is 

\[ \gamma r = F_{d,term}
\]

where \(r\) is some characteristic length scale describing the largest raindrop. Determining \(F_{d,term}\) is potentially subtle, since the pre-collapse deformation of the raindrop is going to certainly change the drag coefficient and the terminal velocity. Let's skip all that and say that the raindrop is always a sphere of radius \(r\) and \(c_D = .5\).




\end{document}