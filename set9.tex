\documentclass[12pt]{article}
\usepackage[utf8]{inputenc}
\usepackage[margin=0.6in]{geometry}
\title{Ph 239 Problem Set 9}
\author{Alex Zahn}
\date{6/2/2016}

\usepackage{amsmath}

\begin{document}

\maketitle

\newcommand{\wmsq}{W/\(\mathrm{m}^2\,\)}
\newcommand{\msq}{\(\mathrm{m}^2\,\)}
\newcommand{\micron}{\(\mu\mathrm{m}\)\,}
\newcommand{\mcb}{\(\mathrm{m}^3\,\)}

\section{Martian Atmosphere}

The gravitational acceleration at the Martian surface is roughly 4 m/s, resulting in a scale height of about 11 km in an all CO\(_2\) atmosphere. The temperature scale height is a factor \(1/(1-1/\gamma)\) larger. For \(\gamma = 1.3\), this comes out to about 47 km. If the Martian surface is at 218 K, the lapse rate is around 4.7 K/km.

At a surface pressure of 636 Pa, the density is .015 kg \mcb. A 70 kg human with a .7 \msq drag area and a drag coefficient of 1.2 has an impressive 210 m/s terminal velocity. If a parachute has a drag coefficient of 2 and a 1000 \msq area, terminal velocity would be 4.3 m/s (just short of 10 mph). So parachuting on mars is a little on the inefficient side.

\section{Wet Air Lapse Rate}

The scale height for temperature is 28 km. At 293 K on the surface, the lapse rate is about 10.5 K/km, and the air reaches 273 K at an altitude of about 1.9 km (closer to 2 km, if we treat the temperature profile as exponential. 2 km is a rounder number, so I'll go with that).

The density of air at 293 K and 50\% relative humidity is 8.5 g/\mcb. A single cubic meter of this has an energy of vaporization of 20 kJ, which would raise the temperature of the same cubic meter of dry air at sea level by 16 K. Spreading out this energy over a total of four cubic meters, we might expect to raise the equilibrium temperature at the cloud altitude by about 4 K.

This might very naively bring the lapse rate down to (20-4) K / 2 km = 8 K/km, yielding a cloud height of 2.5 km.


\section{Carbon Sequestration}

Let's put a lower bound on the problem by optimistically assuming all of our fossil fuels are methane. The average American uses 8 kW of fossil fuels. If the age structure of the US population is stable, we can multiply this by the 79 year life expectancy for a lifetime energy consumption of about 20 TJ.

That's actually a pretty impressive number, considering that a typical nuclear power plant has only has a gigawatt output. A single pathetic mortal's energy habit totals five and half hours of the entire output of an enormous nuclear facility. Really makes you feel like a supervillain.

Methane has a heat of combustion of 55.7 kJ/g and a molecular mass of \(2.66 \times 10^{-23}\) g. So each methane molecule is worth \(1.5 \times 10^{-21}\) kJ.




\section{Everything Wrong with \textit{The Martian}}

It really frustrates me that this movie is considered realistic, and I've been waiting for an opportunity to propose debunking it.

\end{document}